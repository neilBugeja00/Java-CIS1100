\documentclass[a4paper]{article} 
\input{head}
\begin{document}
\graphicspath{{home/Neil/Year 1/Semester 1/Java/Documentation/Images}}

%-------------------------------
%	TITLE SECTION
%-------------------------------

\fancyhead[C]{}
\hrule \medskip % Upper rule
\begin{minipage}{0.295\textwidth} 
\raggedright
\footnotesize
NEIL BUGEJA\hfill\\   
51000L\hfill\\
\end{minipage}
\begin{minipage}{0.4\textwidth} 
\centering 
\large 
JAVA ASSIGNMENT\\ 
\normalsize 
Machine Learning 1, 17/18\\ 
\end{minipage}
\begin{minipage}{0.295\textwidth} 
\raggedleft
\today\hfill\\
\end{minipage}
\medskip\hrule 
\bigskip

%-------------------------------
%	CONTENTS
%-------------------------------
\section{AnyClass class}
\blindtext
\subsection{First Subtask}

\bigskip

\section{Heterogeneous Circular Queue}

\subsection{Construction of CQueue}
In this section a circular queue of size 20 nodes must be created. An integer called 'CIRCULAR\_QUEUE\_SIZE' is created and given the value 20. This is done so that the queue's size is stored in one location and should it need to be changed this can easily be done. \\
\\Next, a for loop is created that repeats for 'CIRCULAR\_QUEUE\_SIZE' times. In every iteration of this loop a new node is created with value null. To create this new node a method addNode is created.\\
addNode will first check whether the node head is null or not. If the head is null it means that the node is the root. However, if this is not the case 

\subsection{Second Subtask}
fuck me jerry

\bigskip

%------------------------------------------------

\section{Second Exercise}
\blindtext
\subsection{First Subtask}

\bigskip

%------------------------------------------------

\end{document}
